%The word �Abstract� should be centered 2? below the top of the page. 
%Skip one line, then center your name followed by the title of the 
%thesis/dissertation. Use as many lines as necessary. Centered below the 
%title include the phrase, in parentheses, �(Under the direction of  
%_________)� and include the name(s) of the dissertation advisor(s).
%Skip one line and begin the content of the abstract. It should be 
%double-spaced and conform to margin guidelines. An abstract should not 
%exceed 150 words for a thesis and 350 words for a dissertation. The 
%latter is a requirement of both the Graduate School and UMI's 
%Dissertation Abstracts International.
%Because your dissertation abstract will be published, please prepare and 
%proofread it carefully. Print all symbols and foreign words clearly and 
%accurately to avoid errors or delays. Make sure that the title given at 
%the top of the abstract has the same wording as the title shown on your 
%title page. Avoid mathematical formulas, diagrams, and other 
%illustrative materials, and only offer the briefest possible description 
%of your thesis/dissertation and a concise summary of its conclusions. Do 
%not include lengthy explanations and opinions.
%The abstract should bear the lower case Roman number ii (if you did not 
%include a copyright page) or iii (if you include a copyright page).

\begin{center}
% \vspace*{2in}
{\large \textbf{ABSTRACT}}
\vspace{11pt}

\begin{singlespace}
\rowaco: \disstitle\\
(Under the direction of \wiva)
\end{singlespace}
\end{center}

Genetic mapping is a process by which researchers seek to identify genetic factors that influence a trait of interest.
Such efforts typically focus on those that either increase or decrease the trait of interest, and assume that the variance of the trait is constant across all individuals.
I develop and apply statistical methods that challenge that assumption in two ways.
First, I consider the situation where non-genetic factors influence trait variance, which I term ``background variance heterogeneity''.
Though they are not of immediate interest in a genetic mapping study, they can be exploited to align observations' weights with their precisions.
Second, I consider the situation where genetic factors influence trait variance, which I term ``foreground variance heterogeneity''.
Such factors are of immediate interest because they represent novel discoveries that could be missed by standard analyses.

I consider both foreground and background variance heterogeneity as they relate to linkage disequilibrium mapping in exchangeable mapping populations.
I report three novel genetic factors with strong evidence that they influence medically-important traits in the mouse model system.
Finally, I consider the background variance heterogeneity as it relates to association mapping in non-exchangeable populations.
I report a mathematical advance that makes possible the fitting of a statistical model that accommodates background variance heterogeneity in non-exchangeable populations.

\clearpage
